\documentclass[a4paper]{article}
\usepackage{kotex}
\usepackage{better_poster}
\newcommand{\latex}{\LaTeX\xspace}
\newcommand{\tex}{\TeX\xspace}

% ---- fill in from here







% authors
\title{전문성 계발을 위한 워크샵을 다음과 같이 개최하오니 관심 있는 분들의 많은 참여를 바랍니다.}
\author{Lana Sinapayen, Someone Else}

% type of poster: [exp]erimental results, [methods], [theory]
% Disclaimer: the original classification had "study" and "intervention" as separate categories. I group them under experimental results.
\newcommand\postertype{exp} % [exp],[methods],[theory]

\begin{document}

	
% main point of your study
\makefinding{
\textbf{The 2nd} 
\textbf{\LaTeX } 
\textbf{Workshop}\\
{\huge {for teachers in Gyeonggi Science Highschool for the Gifted.}}
}

% the main text of your poster goes here

\makemain{
    \section{대상 및 강의 내용}
		\begin{compactitem}
			\item 경기과학고등학교 교직원 중 (학생은 참석 불가함.)\\
				\latex을 이용하여 수업 자료를 만들거나
				학생들의 자율연구, 졸업논문 지도시 \latex을 활용하고 싶으신 분을 대상으로,
			\item \href{http://bfstudy.net/}{bfstudy.net} 회원 가입 및  \href{https://moodle.bfstudy.net/course/view.php?id=79}{경기과학고 \tex 사용자 협회 무들 강좌}를 자기주도적으로 학습할 수 있도록 도와 드립니다.
		\end{compactitem}

    \section{일시 및 장소}
		\begin{compactitem}
	    	\item 2021년 4월 22일 14:00 - 16:30
			\item 과학영재연구센터(SRC) 629호
		\end{compactitem}

		\begin{tikzpicture}
			\node[anchor=south west] (image) at (0,0) {\includegraphics[width=9cm]{167582351_10223958451496533_4019638851113294173_n}};
			\draw[latex-, very thick,green] (2.5,3.2) -- ++(-0.5,0) node[left,black,fill=yellow]{\small Here is the SRC629!!};

		\end{tikzpicture}    
	  
	\section{참여 방법 및 준비물}
		\begin{compactitem}
       		\item 아래 QR 코드 스캔 또는 \href{https://forms.gle/9jCNwDB5NPxRsF2t8}{여기를 클릭}하여 신청해 주세요. 
       		(참가 인원 파악을 위해 꼭 신청해 주시기 바랍니다.)
    		\item 개인 랩탑 지참
    		(가급적 \latex\을 설치해 오시기 바랍니다.)
    		\item \href{http://bfstudy.net/}{bfstudy.net} 회원 가입 및  \href{https://moodle.bfstudy.net/course/view.php?id=79}{경기과학고 \tex 사용자 협회 무들 강좌} 예습 권장	
      	\end{compactitem}
}
% If you have extra figures or data to show
\makeextracolumn{}

% footer
% generate qr code from https://www.qr-code-generator.com/ and replace qr_code.png
% default: barcode on the left
%\makefooter{MARK-2013.pdf}{images/qr-code.png}

\vspace*{\fill}
\noindent\makebox[\linewidth][c]{%
	\colorbox{\postertype}{%
		\parbox[c][0.14\paperheight][c]{\paperwidth}{%
			\hspace*{\dimexpr\hoffset+\oddsidemargin+1in\relax}%
			\begin{minipage}[c][0.12\paperheight][c]{0.8\textwidth}
				\href{https://www.gs.hs.kr}
				{\includegraphics[width=0.9\textwidth]{logo_name-1.jpg}}			

			\end{minipage}
			\hfill
			\begin{minipage}[c][0.14\paperheight][c]{0.20\textwidth}
				\href{https://forms.gle/9jCNwDB5NPxRsF2t8}{
				\includegraphics[height=0.12\paperheight]{images/qr-code.png}}
			\end{minipage}
		}%
	}%
	}\par\medskip


% replace with this like for barcode on the right
%\makealtfooter{images/uni_logo.png}{images/qr-code.png}
 
\end{document}